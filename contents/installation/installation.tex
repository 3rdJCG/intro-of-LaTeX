%!TEX encoding = UTF-8
% +++
% latex = "uplatex"
% +++
\expandafter\ifx\csname ifdraft\endcsname\relax
    \input{../../settings.tex}
    \graphicspath{{./figure/}}
\begin{document}
\fi

\chapter{環境構築と初期設定}
	\LaTeX を使うにあたり最も大きな障壁とされるのが環境構築\footnote{そのソフトを使える環境を整えること}と言われています。
	確かにMicrosoft Wordなどと比べれば導入は少々煩雑ではありますが、
	多くの方々の尽力により今ではとても簡単になっているので、身構えることはありません。

	\section{\LaTeX を使うのに必要なもの}
		まず、\LaTeX のソフトそのものが必要になりますが、\LaTeX は単体のソフトウェアではなく、多くの関連ソフトの集合体です。
		それらを一つ一つ導入していくのはとても手間がかかるので、
		\LaTeX では関連するソフトをひとまとまりにした状態で配布するディストリビューションという形態がとられています。
		現在配布されているディストリビューションにも数種類あるのですが、
		現在最もポピュラーな\TeX Liveというディストリビューションを今回使用します。

		また\LaTeX は、マークアップ言語と呼ばれるプログラミング言語のような形で文章の構造を指定します。
		そのため\LaTeX を使うためには\LaTeX 本体ソフトウェア以外にもマークアップ言語を記述するためのテキストエディタが必要になります。
		\TeX Liveにも一応\TeX Worksというテキストエディタが同梱されているのですが、お世辞にもモダンとは言えません。
		ですので、今現在\LaTeX に限らず多くのプログラミング言語の開発環境に用いられているVisual Studio Codeというエディタを使用します。

		今回インストールするソフトウェアは以下の2つとなります。
		\begin{description}
			\item[\LaTeX ディストリビューション] \TeX Live
			\item[テキストエディタ] Visual Studio Code
		\end{description}
		これらのソフトウェアの導入手順を以下にて解説します。

	\section{\TeX Liveの導入}
		まず最初に\TeX Liveを導入します。

		\subsection{Windows}
			以下のページよりネットワークインストーラをダウンロードします。
			\begin{mdframed}[style=shadow]
				\url{https://www.tug.org/texlive/acquire-netinstall.html}
			\end{mdframed}
			このページを開いて、\url{install-tl-windows.exe}をダウンロードします。

			\begin{wrapfigure}{r}[0pt]{0.5\textwidth}
				\centering
				\includegraphics[width=5cm]{TeXlive-installer2.png}
			\end{wrapfigure}

			ダウンロードしたファイルを起動すると展開用のソフトが起動するので、installを選択し次に進み、installボタンを押下します。
			すると右のようなインストーラが起動します。
			ここで様々なインストールの設定が行えますが、たいていの場合変更は必要ありません。
			インストール先もここの設定で変更が可能ですが、本書では変更しないという前提で解説をします。

			インストールボタンを押下するとインストールが開始されます。
			かなり時間がかかるので放置しておきます。

			終わったらインストーラを閉じてインストールは完了です。

		\subsection{Linux}
			\subsubsection{Debian,Ubuntuなど}
				パッケージ管理ソフトを使用するのでまず最初に更新を行います。
				以下のようにしてパッケージとリストを更新します。
				\begin{mdframed}[style=bash]
					\begin{verbatim}
					$ sudo apt update
					$ sudo apt upgrade
					\end{verbatim}
				\end{mdframed}
				更新したら、以下のようにしてTeX Liveと関連ソフトウェアをインストールします。
				\begin{mdframed}[style=bash]
					\begin{verbatim}
					$ sudo apt install texlive-lang-japanese ghostscript perl
					$ sudo apt install evince poppler-utils poppler-data
					$ sudo apt install texlive-fonts-recommended texlive-fonts-extra
					\end{verbatim}
				\end{mdframed}
				以上で完了です。

			\subsubsection{Arch Linuxなど}
				Ubuntuなどと同様にパッケージリストを更新します。
				\begin{mdframed}[style=bash]
					\begin{verbatim}
					$ sudo pacman -Syu
					\end{verbatim}
				\end{mdframed}
				更新したら、同様に以下のようにしてTeX Liveと関連ソフトウェアをインストールします。
				\begin{mdframed}[style=bash]
					\begin{verbatim}
					$ sudo pacman -S texlive-langjapanese texlive-most
					$ sudo pacman -S ghostscript perl
					$ sudo pacman -S evince poppler-utils poppler-data
					\end{verbatim}
				\end{mdframed}
				以上で完了です。

	\section{VSCodeの導入}
		次にMicrosoft Visual Studio Code(以後VSCodeと省略)を導入します。

		\subsection{Windows}
			以下のページよりインストーラをダウンロードします。
			\begin{mdframed}[style=shadow]
				\url{https://code.visualstudio.com/}
			\end{mdframed}
			インストーラを起動して、指示にしたがってインストールを進めます。

			以下のようなチェックボックスのある画面が出たら、
			\begin{itemize}
				\item ファイルコンテキストメニューに[Codeで開く]アクションを追加\footnote[1]{この項目は必須ではありませんが便利なのでおすすめです}
				\item ディレクトリコンテキストメニューに[Codeで開く]アクションを追加\footnotemark[1]
				\item PATHへの追加
			\end{itemize}
			のチェックボックスにチェックを入れて次へを押下します。
			その後は指示にしたがってインストールを進めたら完了です。

		\subsection{Linux}
			\subsubsection{Debian,Ubuntuなど}
				以下のページより.deb形式のファイルをダウンロードします。
				\footnote{コマンドラインのみでのインストールも可能ですがここでは省略します。
					\\詳細は\url{https://code.visualstudio.com/docs/setup/linux}を参照してください。}
				\begin{mdframed}[style=shadow]
					\url{https://code.visualstudio.com/}
				\end{mdframed}
				以下のコマンドを実行してダウンロードした.deb形式のファイルからインストールします。
				ファイルパスはダウンロードしたファイル名に適宜変更します。
				\begin{mdframed}[style=bash]
					\begin{verbatim}
					$ sudo apt install ./<file>.deb
					\end{verbatim}
				\end{mdframed}
				以上で完了です。

			\subsubsection{Arch Linuxなど}
				以下のコマンドを実行してパッケージ管理ソフト経由でインストールします。
				\footnote{このリポジトリはOSS版であり、Microsoft公式のリポジトリではないようです。
					\\使用に関して全く問題はありませんが、公式のリポジトリからインストールする場合は
					\\\url{https://code.visualstudio.com/docs/setup/linux}を参照してください。}
				\begin{mdframed}[style=bash]
					\begin{verbatim}
					$ sudo pacman -S code
					\end{verbatim}
				\end{mdframed}
				以上で完了です。

	\section{VSCodeの設定}
		\subsection{日本語環境のセットアップ}
			前項でVSCodeのインストールを行いましたが、インストールしたままの状態だと\LaTeX の執筆があまりしやすいとは言えません。
			そのため本項では、\LaTeX の執筆がしやすいようにVSCodeの設定を変更していきます。

			インストールしたままの状態では言語環境が英語なので、日本語の環境に設定します。
			VSCodeを起動し、キーボードで~
			\tikzmarkin[]{1}(0.1,0.4)(-0.1,-0.15)Ctrl\tikzmarkend{1}~+
			\tikzmarkin[]{2}(0.1,0.4)(-0.1,-0.15)Shift\tikzmarkend{2}~+
			\tikzmarkin[]{3}(0.1,0.4)(-0.1,-0.15)X\tikzmarkend{3}
			を押下します。
			拡張機能のタブが開かれるので、検索バーにJapanと入力します。
			すると候補にJapanese Language Pack for Visual Studio Codeが表示されるので、インストールのボタンを押下します。
			\begin{figure}[H]
				\centering
				\includegraphics[width=120mm,trim=0 300 0 0,clip]{VSCode-lang-ja.png}
			\end{figure}
			インストールが完了すると以下のようにリロードを要求されるので、Yesを押下してリロードを行い日本語環境の設定は完了です。
			\begin{figure}[H]
				\centering
				\includegraphics[width=80mm,trim=420 10 0 435,clip]{VSCode-lang-ja-installed.png}
			\end{figure}

		\subsection{LaTeX Workshopの導入}
			次に\LaTeX の執筆を支援する拡張機能であるLaTeX Workshopを導入します。
			先ほどと同様に~
			\tikzmarkin[]{4}(0.1,0.4)(-0.1,-0.15)Ctrl\tikzmarkend{4}~+
			\tikzmarkin[]{5}(0.1,0.4)(-0.1,-0.15)Shift\tikzmarkend{5}~+
			\tikzmarkin[]{6}(0.1,0.4)(-0.1,-0.15)X\tikzmarkend{6}
			を押下し拡張機能のタブを開き、検索バーにLaTeXと入力します。
			候補の中からLaTeX Workshopを選択し、インストールを押下します。
			\begin{figure}[H]
				\centering
				\includegraphics[width=120mm,trim=0 300 0 0,clip]{VSCode-latexworkshop.png}
			\end{figure}
			しばらくするとインストールが終わった旨の表示がされ、導入は完了となります。

		\subsection{VSCodeの設定}
			\begin{wrapfigure}{r}[0pt]{0.5\textwidth}
				\centering
				\includegraphics[width=60mm,trim=0 30 50 0,clip]{VSCode-setting-dialog.png}

				\vspace{2.5mm}
				
				\includegraphics[width=60mm]{VSCode-setting-json.png}
			\end{wrapfigure}
			執筆に当たり必要な拡張機能は入れ終わったので、設定をしていきます。
			まず、
			\tikzmarkin[]{7}(0.1,0.4)(-0.1,-0.15)Ctrl\tikzmarkend{7}~+
			\tikzmarkin[]{8}(0.1,0.4)(-0.1,-0.15)Shift\tikzmarkend{8}~+
			\tikzmarkin[]{9}(0.1,0.4)(-0.1,-0.15)P\tikzmarkend{9}
			を押下して、settingsと入力し、候補に出てきた『基本設定: 設定(JSON)を開く』を選択して開きます。
			すると図のように\url{"settings.json"}が開かれることを確認します。
			次に、以下のページから設定ファイルをコピーします。
			\begin{mdframed}[style=shadow]
				\url{https://gist.github.com/}
			\end{mdframed}
			コピーした設定ファイルを先ほど開いた\url{"settings.json"}の\{~\}の中に貼り付けます。
			貼り付けた後、~
			\tikzmarkin[]{10}(0.1,0.4)(-0.1,-0.15)Ctrl\tikzmarkend{10}~+
			\tikzmarkin[]{12}(0.1,0.4)(-0.1,-0.15)S\tikzmarkend{12}
			で保存して設定は完了です。

		\subsection{動作確認}
			これまでの作業で正しく環境構築と初期設定が行われているか、実際に\LaTeX 文書を作ってみて確認してみましょう。

			\expandafter\ifx\csname ifdraft\endcsname\relax
\end{document}
\fi