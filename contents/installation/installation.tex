\expandafter\ifx\csname ifdraft\endcsname\relax
    \input{../../settings.tex}
    \graphicspath{{./figure/}}
\begin{document}
\fi

\chapter{環境構築と初期設定}
	\LaTeX を使うにあたり最も大きな障壁とされるのが環境構築\footnote{そのソフトを使える環境を整えること}と言われています。
	確かにMicrosoft Wordなどと比べれば導入は少々煩雑ではありますが、
	多くの方々の尽力により今ではとても簡単になっているので、身構えることはありません。

	\section{\LaTeX を使うのに必要なもの}
		まず、\LaTeX のソフトそのものが必要になりますが、\LaTeX は単体のソフトウェアではなく、多くの関連ソフトの集合体です。
		それらを一つ一つ導入していくのはとても手間がかかるので、
		\LaTeX では関連するソフトをひとまとまりにした状態で配布するディストリビューションという形態がとられています。
		現在配布されているディストリビューションにも数種類あるのですが、
		現在最もポピュラーな\TeX Liveというディストリビューションを今回使用します。

		また\LaTeX は、マークアップ言語と呼ばれるプログラミング言語のような形で文章の構造を指定します。
		そのため\LaTeX を使うためには\LaTeX 本体ソフトウェア以外にもマークアップ言語を記述するためのテキストエディタが必要になります。
		\TeX Liveにも一応\TeX Worksというテキストエディタが同梱されているのですが、お世辞にもモダンとは言えません。
		ですので、今現在\LaTeX に限らず多くのプログラミング言語の開発環境に用いられているVisual Studio Codeというエディタを使用します。

		今回インストールするソフトウェアは以下の2つとなります。
		\begin{description}
			\item[\LaTeX ディストリビューション] \TeX Live
			\item[テキストエディタ] Visual Studio Code
		\end{description}
		これらのソフトウェアの導入手順を以下にて解説します。

	\section{\TeX Liveの導入}
		まず最初に\TeX Liveを導入します。

		\subsection{Windows}
			以下のページよりネットワークインストーラをダウンロードします。	\\
			\url{https://www.tug.org/texlive/acquire-netinstall.html}	\\
			このページを開いて、\url{install-tl-windows.exe}をダウンロードします。

			\begin{wrapfigure}{r}[0pt]{0.5\textwidth}
				\centering
				\includegraphics[width=5cm]{TeXlive-installer2.png}
			\end{wrapfigure}

			ダウンロードしたファイルを起動すると展開用のソフトが起動するので、installを選択し次に進み、installボタンを押下します。
			すると右のようなインストーラが起動します。
			ここで様々なインストールの設定が行えますが、たいていの場合変更は必要ありません。
			インストール先もここの設定で変更が可能ですが、本書では変更しないという前提で解説をします。

			インストールボタンを押下するとインストールが開始されます。
			かなり時間がかかるので放置しておきます。

			終わったらインストーラを閉じてインストールは完了です。

		\subsection{Linux}
			\subsubsection{Debian,Ubuntuなど}
				パッケージ管理ソフトを使用するのでまず最初に更新を行います。
				以下のようにしてパッケージとリストを更新します。
				\begin{mdframed}[style=bash]
					\begin{verbatim}
					$ sudo apt update
					$ sudo apt upgrade
					\end{verbatim}
				\end{mdframed}
				更新したら、以下のようにしてTeX Liveと関連ソフトウェアをインストールします。
				\begin{mdframed}[style=bash]
					\begin{verbatim}
					$ sudo apt install texlive-lang-japanese ghostscript perl
					$ sudo apt install evince poppler-utils poppler-data
					$ sudo apt install texlive-fonts-recommended texlive-fonts-extra
					\end{verbatim}
				\end{mdframed}
				以上で完了です。

			\subsubsection{Arch Linuxなど}
				Ubuntuなどと同様にパッケージリストを更新します。
				\begin{mdframed}[style=bash]
					\begin{verbatim}
					$ sudo pacman -Syu
					\end{verbatim}
				\end{mdframed}
				更新したら、同様に以下のようにしてTeX Liveと関連ソフトウェアをインストールします。
				\begin{mdframed}[style=bash]
					\begin{verbatim}
					$ sudo pacman -S texlive-langjapanese texlive-most
					$ sudo pacman -S ghostscript perl
					$ sudo pacman -S evince poppler-utils poppler-data
					\end{verbatim}
				\end{mdframed}
				以上で完了です。
		
		\section{VSCodeの導入}
			次にMicrosoft Visual Studio Code(以後VSCodeと省略)を導入します。

			\subsection{Windows}
				以下のページよりインストーラをダウンロードします。
				\begin{mdframed}[style=shadow]
					\url{https://code.visualstudio.com/}
				\end{mdframed}
				インストーラを起動して、指示にしたがってインストールを進めます。
				以下のようなチェックボックスのある画面が出たら、
				\begin{itemize}
					\item ファイルコンテキストメニューに[Codeで開く]アクションを追加
					\item ディレクトリコンテキストメニューに[Codeで開く]アクションを追加
					\item PATHへの追加
				\end{itemize}
				のチェックボックスにチェックを入れて次へを押下します。
				その後は指示にしたがってインストールを進めたら完了です。
				
				

				\expandafter\ifx\csname ifdraft\endcsname\relax
\end{document}
\fi