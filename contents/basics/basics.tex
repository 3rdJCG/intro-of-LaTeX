%!TEX encoding = UTF-8
% +++
% latex = "uplatex"
% +++
\expandafter\ifx\csname ifdraft\endcsname\relax
    \input{../../settings.tex}
    \graphicspath{{./figure/}}
\begin{document}
\fi

\chapter{\logoLaTeX の基本文法}
    前章で\LaTeX の環境構築と初期設定が終わりました。
    少々長かったですが、これで\LaTeX を使用する準備は万端です。
    本章では、構築した環境を利用して\LaTeX の基本的な機能を使った文書を作っていきます。

    \section{\LaTeX 文書の基本}
        まず、基本的な\LaTeX 文書の構成について解説します。
        \LaTeX 文書の基本的な例を以下に示します。
        \begin{minted}[linenos,fontsize=\small,highlightlines={1-6}]{tex}
\documentclass[uplatex,dvipdfmx]{jsarticle}

\title{テスト文書}
\author{わたし}
\date{\today}

\begin{document}
    \maketitle

    \section{セクション}
        セクションです。

        \subsection{サブセクション}
            サブセクションです。

\end{document}
        \end{minted}
        \LaTeX の構文では、大きく分けて2種類の記述があります。
        1行目や3行目のように\mintinline{tex}{\}で始まるコマンドという記述と、
        11行目や14行目のように\mintinline{tex}{\}で始まらない文章があります。
        文章は書いた内容がそのまま出力させる指示になりますが、
        コマンドの場合は\LaTeX ソフトウェアに特殊な指示を与えます。

        また\LaTeX 文書は、その文書の設定について記述するプリアンブルと、
        文書そのものを記述する本文の2つで構成されます。
        たとえばこの文書では、ハイライトされている1行目から6行目がプリアンブルにあたります。
        プリアンブルでは通常の文章は記述できません。
        ハイライトされていない\mintinline{tex}{\begin{document}}で始まり
        \mintinline{tex}{\end{document}}で終わる部分が本文にあたります。

        \subsection{プリアンブルの内容}
            \mintinline{tex}{\documentclass}ではその文書のタイプセット手順、
            言語設定やレイアウトなどについて設定します。
            今回の場合日本語環境で最も使われる\LaTeX エンジンのuplatexを使用するよう指定し、
            後処理にdvipdfmxというソフトを使用するように指定しています。
            また文書のレイアウトを日本語環境レポート向けのjsarticleに指定しています。
            このコマンドは\LaTeX 文書において必ず記述する必要があります。

            \mintinline{tex}{\title}では文書のタイトルを、
            \mintinline{tex}{\author}では文書の作成者を、
            \mintinline{tex}{\date}では文書の作成日など、
            ここではタイトルに記述する内容について設定しています。
            しかし、タイトルの出力自体はこの記述ではされません。
            本文中の\mintinline{tex}{\maketitle}を記述することによってタイトルが出力されます。

        \subsection{本文の内容}
            \mintinline{tex}{\maketitle}ではプリアンブルで設定したタイトル・作成者・作成日を出力します。
            \mintinline{tex}{\maketitle}を記述しないとタイトルが作成されず、いきなり本文から始めることができます。
            
            \mintinline{tex}{\section}では文書の節を出力します。
            このコマンドは記述した箇所から次に\mintinline{tex}{\section}
            が記述されるまでの文章がその節の区間となります。
            つまりこの文書では10行目から12行目までに記述された内容が第1節となります。

            \mintinline{tex}{\subsection}では文書の小節を出力します。
            \mintinline{tex}{\section}より1つレベルが低くなり、
            第1節の範囲内で\mintinline{tex}{\subsection}を使用すると第1.1節といったように出力されます。
            同様のコマンドに\mintinline{tex}{\subsubsection}なども存在します。

    \section{数式の書き方}
        レポートを執筆するにあたり必須の機能である数式出力をしてましょう。
        まずは
        

            \expandafter\ifx\csname ifdraft\endcsname\relax
\end{document}
\fi