%!TEX encoding = UTF-8
% +++
% latex = "uplatex"
% +++
\expandafter\ifx\csname ifdraft\endcsname\relax
    \input{../../settings.tex}
    \graphicspath{{./figure/}}
\begin{document}
\fi

\chapter{\logoLaTeX の基本文法}
    前章で\LaTeX の環境構築と初期設定が終わりました。
    少々長かったですが、これで\LaTeX を使用する準備は万端です。
    本章では、構築した環境を利用して\LaTeX の基本的な機能を使った文書を作っていきます。

    \section{\LaTeX 文書の基本構成}
        まず、基本的な\LaTeX 文書の構成について解説します。
        \LaTeX 文書の基本的な例を以下に示します。
        
        \begin{minted}[linenos,fontsize=\small,highlightlines={7-16}]{tex}
\documentclass[uplatex,dvipdfmx]{jsarticle}

\title{テスト文書}
\author{わたし}
\date{\today}

\begin{document}
    \maketitle

    \section{セクション}
        セクションです。

        \subsection{サブセクション}
            サブセクションです。

\end{document}
        \end{minted}

        % \LaTeX 文書は、基本的にその文書の設定について記述するプリアンブルと、
        % 文書そのものを記述する本文の2つで構成されます。

            \expandafter\ifx\csname ifdraft\endcsname\relax
\end{document}
\fi